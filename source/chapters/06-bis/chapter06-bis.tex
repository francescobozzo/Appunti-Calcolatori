\documentclass[class=book, crop=false]{standalone}
\usepackage[subpreambles=true]{standalone}

\usepackage{../../style}

%\graphicspath{{./assets/images/}}%

\begin{document}
\chapter{L'architettura MIPS}
\section{Gestione delle procedure}
L'utilità dei calcolatori sarebbe molto limitata se questi non avessero la possibilità di svolgere delle procedure, che possono essere immaginate a tutti gli effetti come delle funzioni che dato un certo input eseguono un determinato task a cui sono dedicate.\\
L'aspetto fondamentale che si deve curare per dare la possibilità ai calcolatori di svolgere procedure è la definizione di un protocollo di chiamata delle procedure che deve stabilire con precisione questi aspetti:
\begin{itemize}[noitemsep]
	\item Come caricare i parametri di input della procedura in posti noti
	\item Come trasferire il controllo alla procedura, che deve:
		\begin{itemize}[nolistsep, noitemsep]
			\item Acquisire le risorse necessarie
			\item Eseguire il task affidatole
			\item Caricare gli output in posti noti
			\item Restituire il controllo al chiamante
		\end{itemize}
	\item Come salvare il valore di ritorno della procedura e "fare pulizia" del tutto
\end{itemize}
I protocolli per la gestione delle procedure sono diversi in base all'architettura che si utilizza ed alle convenzioni di chiamata del compilatore, noi per ora studiamo il caso del MIPS.

\section{Protocollo di chiamata MIPS}
L'idea fondante del protocollo del MIPS è di utilizzare gniqualvolta possibile i registri, dato che essi sono il meccanismo più veloce a disposizione per la gestione dei parametri delle procedure. Ecco una tabella riassuntiva delle convensioni sui registri definite nel MIPS:
%\begin{table}[H]
%	\centering
%	\caption{didascalia}
%	\subimport{assets/tables/}{convenzioni-registri-MIPS.tex}
%\end{table}



\end{document}
