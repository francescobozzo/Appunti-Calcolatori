\documentclass[class=book, crop=false]{standalone}
\usepackage[subpreambles=true]{standalone}

\usepackage{../../style}

\graphicspath{{./assets/images/}}

\begin{document}
\chapter{Architettura ARM}
\paragraph{L'architettura ARM} nasce negli anni '80 come architettura RISC leggermente pragmatica, che tenta di migliorare la suddetta architettura implementando delle istruzioni e delle strategie appartenenti ad Intel, probabilmente è esattamente questa la ragione del suo successo.\\
Un'altra caratteristica peculiare dell'ARM è la presenza di una modalità \emph{low power}, che limita di molto il consumo di energia restringendo le istruzioni utilizzabili solo a quelle più semplici (e quindi meno costose). È propio per questa ragione che l'ARM ha avuto un'enorme diffusione come architettura per i dispositivi \emph{portable} e \emph{wearable}.

\section{Registri e relativo utilizzo}

\paragraph{Le convenzioni di chiamata}

\section{Modalità di indirizzamento in ARM}

\section{Particolarità delle istruzioni ARM}	 


\end{document}
