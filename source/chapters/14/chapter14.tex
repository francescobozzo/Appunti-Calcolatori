\documentclass[class=book, crop=false, oneside]{standalone}
\usepackage[subpreambles=true]{standalone}

\usepackage{../../style}

\graphicspath{{./assets/images/}}

% arara: pdflatex: { synctex: yes, shell: yes }
% arara: latexmk: { clean: partial }
%! arara: clean: { extensions: [sta] }
\begin{document}
\chapter{I/O}

\section{La necessità di comunicare}
Un calcolatore sarebbe praticamnte inutile se non avesse la possibilità di interfacciarsi e comunicare con l'esterno, ed è proprio di questo che si occupano le periferiche di input/output (da qui in avanti I/O); Questi dispositivi possono essere estremamente diversi tra loro a seconda del compito che sono chiamati a svolgere, tuttavia devono avere delle caratteristiche comuni, ovvero devono essere \emph{espandibili} ed \emph{eterogenei}.

% TODO chiarisci cosa significa di preciso che devono essere eterogeneri


\subsection*{Tre termini tecnici}
Chiariamo ora questi termini tecnici definendone precisamente il significato poiché ci torneranno utili nel resto del capitolo.
\paragraph{Transizione di I/O} Invio indirizzo e spedizione o ricezione dei dati.
\paragraph{Input} Trasferimento di dati da una periferica verso la memoria dove il processore può leggerla.
\paragraph{Output} Trasferimento di dati dalla memoria ad un dispositivo.

Data l'enorme varietà di compiti che assolvono le periferiche di I/O a  seconda del tipo di applicazione, posso essere interessato a diverse prestazioni:
\begin{itemize}
	\item in alcuni casi ad la cosa più importante è il tempo di accesso/risposta (latenza), ad esempio quando si tratta di tasitere o mouse;
	\item in altri casi la caratteristica di spicco deve essere il troughput, vedi i sistemi di streaming;
	\item infine esistono molti altri casi meno comuni che necessitano di caratteristiche dedicate, ad esempio un sistema bancario può avere la necessità di massimizzare il numero di file di piccole dimensioni su cui opera contemporaneamente.
\end{itemize}

Dati questi esempi si possono introdurre le principali caratteristiche secondo cui vengono suddivise le suddette periferiche: operazioni possibili (R e/o W), \emph{partner} (uomo o macchina) e velocità di trasferimento; in seguito viene riportata un tabella riassuntiva delle maggiori classi di dispositivi di questo tipo.

\begin{figure}[H]
	\centering
	\includegraphics[width=0.9\textwidth,keepaspectratio]{classificazione-periferiche}
	\caption{Classificazione delle periferiche}
\end{figure}

\section{Connesione tra processore e periferiche}
Nonostante le differenze di cui si è parlato nella sezione precedente tutte le periferiche hanno una caratteristica comune, ovvero la modalità di collegamento al processore.

Tutte le connessioni avvengono attraverso strutture di comunicazione dette \emph{bus}, di cui esistono due tipologie:
\begin{itemize}
	\item bus processore/memoria: sono specializzati, corti e veloci;
	\item bus I/O: sono utilizzati per la comunicazione con periferiche generiche, possono essere relativamente lunghi e comunque non si interfacciano direttamente con la memoria, ma richiedono come intermediario un bus processore/memoria o un bus di sistema.
\end{itemize}
Nelle prime semplici architetture avevamo un unico grosso bus parallelo che collegava tutte le componenti, ma per problemi di clock e frequenze ora si usano architetture di comunicazione più complesse fatte di più bus paralleli condivisi e di bus seriali punto/punto.
% TODO che cosa significa bus di sistema? è un terzo tipo di bus? e cosa sono i bus seriali punto a punto?? inoltre un singolo bus parallelo è parallelo a se stesso?
Saranno ora discusse due implementazioni diverse del bus che convivono nei nostri calcolatori: sincrona ed asincrona.

\subsection{Bus sincrono}
Si tratta di un bus attraverso cui le informazioni vengono trasmesse in modo sincrono, quindi le cmunicazioni sono necessariamente scandite dal segnale di clock, che quindi viene trasmesso in una linea di controllo del bus stesso.
% TODO Esempio?? chiarisci esempio pag. 9 grazzz

\begin{figure}[H]
	\centering
	\includegraphics[width=0.9\textwidth,keepaspectratio]{bus-sincrono}
	\caption{Schema rappresentativo del ciclo del bus sincrono}
\end{figure}

Riassumendo quindi i punti di forza del bus di tipo sincrono si deve osservare che questo sistema è tanto semplice da implementare quanto veloce, data la presenza esigua di sequenze di controllo, che quindi non appesantisce lo svolgimento delle operazioni di trasmissione dati.

Di contro si può argomentare che è un sistema che dimostra poca robustezza al \emph{drift} del clock; inoltre utilizzando un sistema simile si costringono tutte le periferiche a lavorare alla stessa velocità del clock, ma questo non è sempre possibile.
% TODO che cosa sarebbe poi questo drift rimane un bel mistero

\subsection{Bus asincrono}
Per ovviare agli sconvenienti appena discussi è stata sviluppata anche una versione asincrona della tecnologia del bus che sostanzialmente non si appoggia più sul clock per il controllo delle transazioni ma utilizza un protocollo in cui le varie fasi sono determinate da degli \emph{handshake}.

Questo cambio di stratagemma libera dalla limitazione del tempo scandito dal clock ma comporta la creazione di apposite linee di controllo per segnalare inizio e fine delle transazioni;
nonostante questa complicazione il bus asincrono garantisce un grande vantaggio, ovvero permette di collegare periferiche a velocità diversa.

\begin{figure}[H]
	\centering
	\includegraphics[width=0.9\textwidth,keepaspectratio]{bus-asincrono}
	\caption{Schema rappresentativo del ciclo del bus asincrono}
\end{figure}

Riassumendo ora i punti di forza del bus asincrono si noti che è un meccanismo che permette di avre un sistema robusto rispetto ai ritardi e che consente di comunicare con periferiche di tipo diverso.
% TODO quali ritardi??

Viceversa osservando i fattori negativi si deve indicare che risulta più lento del bus sincrono data la necessità di inviare e ricevere diversi segnali di controllo ed inoltre, aumentando la complessità del trasferimento I/O, aumenta anche la complessità della circuiteria che sta alla base.

In conclusione dando uno sguardo alla realtà si scopre che spesso si usano tecnologie ibride di bus, in cui c’è un segnale di clock ma che rimangono prevalentemente asincrone.

\begin{figure}[H]
	\centering
	\includegraphics[width=0.9\textwidth,keepaspectratio]{tecnologie-asincrone}
	\caption{Principali tecnologie asincrone}
\end{figure}

% TODO ma qui lo meto l'esempio??

\section{Gestione delle periferiche da parte del SO}
Ora che abbiamo chiarito qual è il tramite fisico attraverso cui i sistemi operativi comunicano con le periferiche rimangono i seguenti interrogativi aperti: come trasformare una richiesta di I/O in un comando per la periferica? E come trasferire i dati?

Di questi problemi si fa carico il sistema operativo (da qui in avanti SO), dato che è la parte di software dedicata alla gestione diretta del processore e delle sue risorse, quindi anche dei sistemi di I/O.

Il SO deve fornire queste funzionalità per la egstione dell'I/O:
\begin{itemize}
	\item garantire che un dato utente abbia accesso ai dispositivi di I/O cui ha i permessi per accedere;
	\item fornire comandi di alto livello per gestire le operazioni di basso livello, trasferimento dati nello specifico;
	\item gestire le interruzioni generate dai dispositivi di I/O (in maniera simile a quanto avviene con le eccezioni generate nei programmi);
	\item ripartire l’accesso a ciascun dispositivo in maniera equa tra i vari programmi che lo richiedono.
\end{itemize}
È chiaro quindi, guardando il terzo punto, che i trasferimenti dati vengono spesso effettuati utilizzando il meccanismo delle interrupt, che hanno un impatto sulle funzionalità del SO e quindi venogno eseguite in una particolare modalità del processore, \emph{supervisor}, cui solo il codice del kernel può accedere.

Inoltre è evidente che per implementare le funzionalità appena elencate il SO deve poter inviare comandi alle periferiche, ricevere notifiche di corretta esecuzione dalle periferiche stesse e consentire trasferimenti diretti tra dispositivi e memoria. Nelle prossime sezioni sono discussi proprio i meccanismi che rendono tutto questo possibile.

\subsection{Come impartire comandi ai dispositivi}
Il sistema operativo impartisce comandi alle varie periferiche fornendo sulle relative linee di bus alcune "parole" di controllo attraverso due metodi possibili:
\begin{itemize}
	\item scrivendo/leggendo in particolari locazioni di memoria (memory mapped I/O);
	\item tramite alcune istruzioni speciali dedicate all’I/O.
\end{itemize}

Per chiarire questo meccanismo conviene procedere tramite un esempio:
scrivendo una particolare parola in una locazione di memoria associata al dispositivo il sistema di memoria ignora la scrittura, mentre il controllore di I/O intercetta l’indirizzo \emph{particolare} e trasmette il dato al dispositivo sotto forma di comando.

Queste particolari locazioni di memoria sono inaccessibili ai programmi utente ma solo il sistema operativo può operarvi, esso viene invocato tarmite una chiamata di sistema, la quale fa commutare il processore in modalità supervisore e rende quindi possibile la scrittura in tali locazioni.
Inoltre la periferica stessa può usare queste locazioni per trasmettere dati o pre-segnalare il suo stato; ad esempio posso chiedere la stampa di un carattere a terminale e a stampa finita un particolare bit di un registro di stato mappato in memoria verrà aggiornato.

\subsection{Come trasmettere/ricevere dati}
Esistono principalmente due meccanismi diferenti per il trasferimento dei dati, il polling e le interruzioni di programma.
Descriviamo in prima battuta il più semplice dei due, il polling.

\section{Polling}
Detta anche modalità di \emph{attesa attiva} sostanzialmente consiste nell'inviare un comando di lettura/scrittura alla periferica e poi iniziare un ciclo di attesa monitorando il bit di stato per sapere quando il dato è pronto.

Di seguito è riportata l'implementazione in assembly MIPS di un ciclo di lettura input da tasitera e in seguito di stampa del dato.

COPIA I CODICI P. 23

Osserviamo ora l'aspetto del costo in termini di risorse del meccanismo del polling con i seguenti esempi:
Consideriamo un processore a 500Mhz e supponiamo che occorrano 400 cicli di clock per un’operazione di polling. Qual è il costo percentuale di questo meccanismo?

\paragraph{Esempio 1: Mouse} Per non perdere movimenti da parte dell’utente occorre acquisire il dato 30 volte al secondo.

Per stimare l'impatto del polling sul processore si deve calcolare il numero di clock al secondo spesi per il polling stesso, ovvero \(30*400 = 12000\) clocks/sec.
Ora che abbiamo il numero di clock spesi per il polling ci basta calcolare a che percentuale di utilizzo del processore corrispondono, ricordando che 500Mhz significa \(500*10^{6}\) clocks/sec.
Quindi complessivamente l'impatto del polling in percentuale è \[12*10^{3}/500*10^{6}=0.002\%\]

È chiaro quindi che in questo caso l'impatto del polling è trascurabile, si noti tuttavia che questo overhead viene pagato sempre, sia che avvenga il trasferimento, sia che non avvenga.

\paragraph{Esempio 2: Hard disk} I dati vengono trasferiti in blocchi di 16 byte ad una velocità di 8MB/s senza la possibilità di perdite.

Il numero di volte al secondo che occorre fare cicli di attesa per non perdere dati è \[\frac{8\textrm{MB/s}}{16\textrm{B}} = 500*10^{3} \textrm{polls/sec}\]
Questo implica che il numero di cicli di clock al secondo dedicati al polling siano \[ 500*10^{3}*400 = 200*10^{6} \textrm{clocks/sec}\]
In percentuale quindi il peso di questo meccanismo è di \(200*10^{6}/500*10^{6}=40\%\) il che è inaccettabile, anche perchè come nel primo esempio questo prezzo in risorse viene pagato sempre.

\subsection{Considerazioni finali sul polling}
L’attesa attiva è un meccanismo che fa perdere tempo al processore dedicando cicli macchina a letture inutili, di conseguenza il polling può essere usato quando le operazioni di I/O avvengono con velocità di trasferimento predeterminata (es. applicazioni di controllo) e comunque il processore ha poco altro da fare.

Sicuramente se i dati vengono trasferiti con elevati \emph{bitrate}\footnote{chissà cosa significa bitrate.} il ciclo di attesa attiva dura poco, in altri casi lo spreco derivante dal polling è inaccettabile e per questo motivo è stato inventato il sistema di I/O a interruzione di programma.

IMPORTA ESEMPIO SPIM??? P. 28

Sì ma segui bene la spiegazione

\section{Input ad interruzione di programma}
Conosciuto anche con il nome di interrupt driven I/O funziona tramite il sollevamento di interruzioni.

Un’interruzione I/O è un segnale usato per indicare al processore che la periferica è pronta ad eseguire il trasferimento richiesto.
Occorre tuttavia un modo per segnalare al processore quale periferica richiede l’interruzione e per gestire il fatto che le interruzioni I/O sono sempre asincrone rispetto all’esecuzione delle istruzioni.

Non esistono particolari istruzioni assembly per eseguire le interruzioni, queste possono presentarsi mentre una qualsiasi istruzione viene eseguita, ma si deve comunque dare modo di terminare l’esecuzione dell’istruzione corrente prima di passare alla gestione dell'interruzione.
In tal senso il programmatore può decidere di posticipare l’esecuzione dell’interruzione a un momento più conveniente programmando sezioni non interrompibili nel codice del kernel, inoltre può classificare le interruzioni secondo grado di priorità.

Il maggiore vantaggio che deriva dall'utilizzo di questa strategia è che non occorre interrompere l’esecuzione del programma se non quando il dato può essere effettivamente riferito in memoria.

Guardando invece il lato negativo occorre un hardware speciale per permettere ai dispositivi di I/O di generare un’interruzione, rilevare l’interruzione, salvare lo stato del processore per eseguire una particolare routine di servizio (Interrupt Service Routine, ISR) e poi riprendere dal punto dove si era interrotto.

\begin{figure}[H]
	\centering
	\includegraphics[width=0.9\textwidth,keepaspectratio]{input-a-interruzione}
	\caption{Meccanismo di input ad interruzione}
\end{figure}

ESEMPIO SPIM?? P.32 sì ma aspetta spiegazione

\subsection{Considerazioni finali sull'interrupt driven I/O}
Nonostante la complessità di questo meccanismo è facile osservare che in caso di trasferimenti di grandi moli di dati esso risulti  molto più efficiente del polling, dimostriamolo riprendendo l'esempio dell'hard disk fato in precedeza:

\paragraph{Esempio 2: Hard disk} Supponiamo che un interrupt costi 500 cicli di clock, poichè è plausibile che costi di più del polling.
Se le interruzioni venissero generate alla frequenza di polling avremmo che \[\frac{\textrm{Disk Interrupts}}{\textrm{sec}}=\frac{8 \textrm{MB/s}}{16\textrm{B}}=500*10^{3}\frac{\textrm{interrupts}}{\textrm{sec}}\] e quindi \[\frac{\textrm{Disk Polling Clocks}}{\textrm{sec}}=500*10^3{3}*500=250*10^{6} \frac{\textrm{clocks}}{\textrm{sec}}\] Di conseguenza la percentuale di utilizzo del processore sarebbe \(250*10^{6}/500*10^{6}= 50\%\). Sembrerebbe che non ci sia guadagno, anzi.

Tuttavia se l’hard disk è attivo solo per il 5\% del tempo, gli interrupt generati saranno il 5\% e la spesa di processore sarà: \(5\% * 50\%=2.5\%\).

Tutto questo proprio grazie al fatto che con il meccanismo di interrupt l’overhead si paga solo quando vengono effettivamente generate richieste.

\section{Eccezioni}
Le interrupts che abbiamo appena visto vanno inserite in una classe più grande di eventi detti eccezioni.

Un eccezione consiste nel trasferimento del controllo del programma per l'avveramento di una condizione appunto eccezionale\footnote{da intendere qui come "fuori dalla norma"}.
Per la gestione di questi eventi il sistema effettua delle azioni specifiche, come registrare il punto di interruzione e salvare lo stato, ed in seguito, a eccezione finita, riprendere dal punto immediatamente successivo al punto di interruzione.

\begin{figure}[H]
	\centering
	\includegraphics[width=0.9\textwidth,keepaspectratio]{schema-eccezioni}
	\caption{Schema proceso delle eccezioni}
\end{figure}

Esistono due tipi di eccezioni:
\paragraph{Interrupts} Sono causate da eventi esterni (I/O) e quindi sono asincrone; possono essere gestite nello spazio tra due istruzioni semplicemente sospendendo il programma e riprendendo, dopo la gestione, dal punto in cui era stato interrotto.
\paragraph{Traps} Sono causate da eventi interni al programma come condizioni eccezionali (es. arithmetic overflow), errori (es. hardware malfunction) o fault (es. non-resident page fault).
Esse sono sincrone all’esecuzione del programma.
Vengono gestite da un \emph{traphandler}, da notare che spesso è possibile riprovare ad eseguire l’istruzione che ha causato l’eccezione o abortire il programma.

\subsection{Supporto del MIPS per la gestione di eccezioni}
La componenete che registra le informazioni necessarie alla gestione delle eccezioni in MIPS è il coprocessore 0.

Quando si presenta un'eccezione in MIPS il registro EPC (registro 14) punta all’indirizzo successivo all’istruzione in esecuzione quando l’eccezione è avvenuta, mentre il registro di stato (registro 12) fa da maschera di abilitazione trap/interrupt.

\begin{figure}[H]
	\centering
	\includegraphics[width=0.9\textwidth,keepaspectratio]{registro-maschera}
	\caption{Registro maschera}
\end{figure}

Il registro BadVAddr (registro 8) contiene l'indirizzo di memoria che ha causato un errore di memoria.
% TODO eventualmente????
Infine il registro cause (registro 13) contiene il tipo dell’eccezione ed i bit pendenti.

\begin{figure}[H]
	\centering
	\includegraphics[width=0.9\textwidth,keepaspectratio]{registro-cause}
	\caption{Registro cause}
\end{figure}

\subsection{Modifiche al processore per gestire le eccezioni}
Per implementare il complesso meccanismo di gestione delle eccezioni il processore deve essere munito di queste nuove componenti:
\begin{itemize}
	\item segnali di controllo per scrivere EPC (EPCWrite), Cause (CauseWrite) e Status;
	\item hardware per registrare il tipo di interruzione in Cause;
	\item modifiche alla macchina a stati in modo che:
	\begin{itemize}
		\item l’indirizzo del gestore dell’interruzione, 8000 0180 hex possa essere caricato in PC (altro ramo nel multiplexer);
		\item Sia salvato l’indirizzo della prossima istruzione da eseguire a ISR terminato in EPC.
	\end{itemize}
\end{itemize}

Ed ecco infine un'immagine del processore con dattapath modificato per supportare la gestione delle eccezioni.

\begin{figure}[H]
	\centering
	\includegraphics[width=0.9\textwidth,keepaspectratio]{final-datapath}
\end{figure}

\end{document}

% TODO inserire le immagini, inserire e descrivere gli esempi SPIM ed utilizzare \unit{}[]
