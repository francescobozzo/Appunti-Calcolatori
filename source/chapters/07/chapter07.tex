\documentclass[class=book, crop=false]{standalone}
\usepackage[subpreambles=true]{standalone}

\usepackage{../../style}

\graphicspath{{./assets/images/}}

\begin{document}

\chapter{L'architettura Intel x86}

\section{Introduzione}
Le CPU \emph{Intel} della famiglia \emph{x86} si basano su un architettura di tipo \emph{CISC}. Esse vengono utilizzate principalmente su laptop, desktop e server, a partire dagli anni Settanta. Un'importante complicazione dell'architettura \emph{Intel} è dettata dal mantenimento della retrocompatibilità: le moderne CPU a 64 bit di ultima generazione sono infatti in grado di eseguire il vecchio codice a 8 bit. A differenza del MIPS infatti le istruzioni \emph{Intel} non sono codificate in una singola parola, bensì possono occupare da 8 a 64 bit.

Come già ampiamente trattato nei precendenti capitoli, un'architettura di tipo \emph{CISC} è in grado di offrire un notevole set di istruzioni e un potente meccanismo di indirizzamento. Ciò implica che non sono più solo le singole operazioni di load e store quelle che permettono di accedere alla memoria.

Per semplicità, in questa dispensa si tratterà l'\emph{ISA} più moderna, chiamata \emph{x86\_64}, nota anche come \emph{amd64}. Essa è caratterizzata da parole a 64 bit e da 16 registri. Fra le varie \emph{ABI} verrà utilizzata quella di Linux, che per qualche aspetto differisce da quella dei sistemi Microsoft e Apple.

\section{La gestione dei registri}



\end{document}
