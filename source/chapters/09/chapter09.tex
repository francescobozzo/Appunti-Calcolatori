\documentclass[class=book, crop=false, oneside]{standalone}
\usepackage[subpreambles=true]{standalone}

\usepackage{../../style}

\graphicspath{{./assets/images/}}
\newmintinline{asm}{}

% arara: pdflatex: { synctex: yes, shell: yes }
% arara: latexmk: { clean: partial }
\begin{document}

\chapter{Esempi e confronti di programmi Assembly}
Ora andiamo a riprendere alcuni esempi già portati nel capitolo dedicato all'Assembly MIPS e andremo a confrontare le implementazioni proposte con le corrispettive per Assembly x86 e ARM.\\
Per completezza, specifichiamo che il codice che mostreremo è stato ottenuto con il compilatore \register{gcc}, lievemente modificato affinché risulti più leggibile (principalmente sull'ordine logico delle istruzioni e sulla nomenclatura dei registri).

\section{Semplici espressioni aritmetico-logiche}
Cominciamo con delle semplici operazioni aritmetiche, che evidenzieranno come le architetture RISC (MIPS e ARM) permettono di eseguire questo tipo di istruzioni in modo molto naturale, al contrario dell'implementazione leggermente più macchinosa di x86.\\
L'espressione è la seguente:
\begin{minted}{c}
f = (g + h) - (i + j);
\end{minted}

\subsubsection{Implementazione MIPS}
In MIPS assumiamo che le variabili \mintinline{c}{g}, \mintinline{c}{h}, \mintinline{c}{i} e \mintinline{c}{j} vengano mappate nei 4 registri dedicati agli argomenti, rispettivamente \register{\$a0}, \register{\$a1}, \register{\$a2} e \register{\$a3}, e decidiamo di salvare il risultato in \register{\$v0}. A questo punto la conversione Assembly è immediata:
\begin{minted}{asm}
add $a0, $a0, $a1  # memorizza in $a0 g + h
add $v0, $a2, $a3  # memorizza in $v0 i + j
sub $v0, $a0, $v0  # memorizza in $v0 (g + h) - (i + j)
\end{minted}

\subsubsection{Implementazione x86}
Siano \mintinline{c}{g}, \mintinline{c}{h}, \mintinline{c}{i} e \mintinline{c}{j} mappate rispettivamente in \register{\%rdi}, \register{\%rsi}, \register{\%rcx} e \register{\%rdx} e si voglia salvare il risultato in \register{\%rax}.\\
Il problema di x86 sta nel fatto che le operazioni sono solo a due operandi, con il secondo che funge da destinazione, e quindi non è possibile specificare un diverso registro di destinazione.\\
Possiamo aggirare questa limitazione sfruttando l'istruzione \mintinline{gas}{lea}: questa infatti ha come argomenti \texttt{<addr>, <dst>} e calcola l'indirizzo \texttt{<addr>} secondo base, spiazzamento e indice shiftato, e salva tutto in \texttt{<dst>}. Sarà quindi sufficiente impostare spiazzamento e shift a 0 per riuscire a sommare agevolmente base (per noi \register{\%rdi}) a indice non shiftato (\register{\%rsi}) e salvarne il risultato in \register{\%rax}.
\begin{minted}{gas}
leaq (%rdi, %rsi), %rax  # %rax = g + h con il trick visto sopra
addq %rcx, %rdx          # %rdx = i + j
subq %rdx, %rax          # %rax = (g + h) - (i + j)
\end{minted}

\subsubsection{Implementazione ARM}
Analogamente a MIPS, anche l'implementazione ARM è molto semplice e naturale, una volta mappate \mintinline{c}{g}, \mintinline{c}{h}, \mintinline{c}{i} e \mintinline{c}{j} in \register{r0}, \register{r1}, \register{r2} e \register{r3} e decidendo di salvare il tutto in \register{r0}.\\
Qui vengono usate \mintinline{asm}{adds} e \mintinline{asm}{subs}, che aggiornano i flag, ma usare le normalissime \mintinline{asm}{add} e \mintinline{asm}{sub} non avrebbe cambiato nulla.
\begin{minted}{arm}
adds r1, r0, r1  ; r1 = g + h
adds r3, r2, r3  ; r3 = i + j
subs r0, r1, r3  ; r0 = (g + h) - (i + j)
\end{minted}

\section{Array e accesso alla memoria}
Ora invece prenderemo come esempio una semplice istruzione che agisce su un array per mostrare diverse modalità di accesso alla memoria: questa volta, x86 permetterà di ridurre notevolmente il numero di righe di codice, mentre le due ISA RISC richiederanno molte più istruzioni.
\begin{minted}{c}
a[12] = h + a[8];
\end{minted}
\subsubsection{Implementazione MIPS}
Come visto in 6.3, in MIPS sarà necessario utilizzare le istruzioni di \mintinline{asm}{lw} e \mintinline{asm}{sw} per accedere all'indirizzo dell'array; inoltre, ricordiamo che  lo shift necessario a ottenere l'indirizzo desiderato andrà sempre moltiplicato per lo spiazzamento di 4 byte, sicché ogni parola risulti lunga 32 bit.\\
Assumendo quindi che \mintinline{c}{h} stia in \register{\$a0} e l'indirizzo dell'array \mintinline{c}{a} in \register{\$a1} otteniamo:
\begin{minted}{asm}
lw $v0, 32($a1)    # carica in $v0 il contenuto di a shiftato di 4 * 8
add $a0, $v0, $a0  # salva in $a0 a[8] + h
sw $a0, 48($a1)    # riponi il contenuto di $a0 nell'indirizzo di a shiftato di 4 * 12
\end{minted}

\subsubsection{Implementazione x86}
La capacità di accedere direttamente alla memoria grazie a istruzioni come \mintinline{gas}{addl} e \mintinline{gas}{movl} semplifica di molto questa operazione per l'ISA Intel.\\
Assumendo che  \mintinline{c}{h} stia \register{\%edi} e l'indirizzo di \mintinline{c}{a} in \register{\%rsi},otteniamo:
\begin{minted}{gas}
addl 32(%rsi), %edi  # %edi = a[8] + h
movl %edi, 48(%rsi)  # sposta a[8] + h in a[12]
\end{minted}

\subsubsection{Implementazione ARM}
La versione ARM è molto simile a quella MIPS.\\
Assumendo che \mintinline{c}{h} stia in \register{r0} e l'indirizzo di \mintinline{c}{a} in \register{r1}, otteniamo:
\begin{minted}{arm}
ldr r3, [r1, #32]  ; carica a[8] in r3
add r0, r3, r0     ; salva in r0 a[8] + h
str r0, [r1, #48]  ; riponi r0 in a[12]
\end{minted}


\section{Blocchi condizionali}
Vediamo ora la gestione di istruzioni condizionali. Solitamente queste vengono gestite da istruzioni di salti condizionati, ma alcune ISA hanno modalità di gestione decisamente peculiari.
\subsection*{Condizioni di uguaglianza}
Andiamo ad analizzare le traduzioni del segente codice C:
\begin{minted}[linenos, breaklines, tabsize = 4]{c}
if (i == j){
	f = g + h;
} else {
	f = g - h;
}
\end{minted}
\subsubsection{Implementazione MIPS}
In MIPS assumiamo che le variabili \mintinline{c}{g}, \mintinline{c}{h}, \mintinline{c}{i} e \mintinline{c}{j} vengano mappate nei 4 registri dedicati agli argomenti, rispettivamente \register{\$a0}, \register{\$a1}, \register{\$a2} e \register{\$a3}, e decidiamo di salvare il risultato in \register{\$v0}.\\
Utilizzeremo quindi \mintinline{asm}{bne} per operare un confronto fra due registri general purpose per verificare \mintinline{c}{i == j}; qualora la condizione non fosse verificata, faremo il salto a L2 e alla sottrazione \mintinline{c}{g -h}. Differentemente, verrà eseguita la somma \mintinline{c}{g + h} e un salto incondizionato \mintinline{asm}{j} farà terminare la sequenza.
\begin{minted}[linenos, tabsize = 4]{asm}
  bne $a2, $a3, L2   # salta a L2 se $a2 e $a3 sono diversi
  add $v0, $a0, $a1
  j L3   # salta a L3
L2:
  sub $v0, $a0, $a1
L3:
\end{minted}

\subsubsection{Implementazione x86}
Siano \mintinline{c}{g}, \mintinline{c}{h}, \mintinline{c}{i} e \mintinline{c}{j} mappate rispettivamente in \register{\%rdi}, \register{\%rsi}, \register{\%rcx} e \register{\%rdx} e si voglia salvare il risultato \mintinline{c}{f} in \register{\%rax}.\\
L'implementazione risulta più complessa rispetto a quella di MIPS, poiché dovremo usare \mintinline{gas}{lea} al posto di \mintinline{gas}{add} e sopratto dovremo affiancare alla \mintinline{gas}{sub} una \mintinline{gas}{mov} per far sì che il risultato vada salvato in \register{\%rax}. Inoltre, è necessaria una \mintinline{gas}{cmpq} per settare i flag e solo dopo possiamo lanciare l'istruzione di salto.
\begin{minted}[linenos, tabsize = 4]{gas}
  cmpq %rcx, %rdx  # confronta i e j e setta il flag
  jne L2   # se i != j, salta a L2
  leaq (%rdi, %rsi), %rax  # salva in %rax g + h
  jmp L3
L2:
  movq %rdi, %rax  # sposta g da %rdi a %rax
  subq %rsi, %rax  # salva in %rax
L3:
\end{minted}
Tutto questo può però essere semplificato dalla magica istruzione \mintinline{gas}{cmov} (\mintinline{gas}{mov} condizionale) di x86:
\begin{minted}[linenos, tabsize = 4]{gas}
leaq (%rdi, %rsi), %rax  # salva in %rax g + h
subq %rsi, %rax  # salva in %rax
cmpq %rcx, %rdx  # i == j
cmovne %rdi, %rax  # sposta %rdi in %rax se il i è diverso da j
\end{minted}

\subsubsection{Implementazione ARM}
L'implementazione ARM ci permette di evitare istruzioni di salto grazie alla sua versione condizionale delle operazioni aritmetiche. Assumendo che \mintinline{c}{g}, \mintinline{c}{h}, \mintinline{c}{i} e \mintinline{c}{j} vengano mappate rispettivamente in \register{r0}, \register{r1}, \register{r2} e \register{r3} e che il risultato vada in \register{r0}, otteniamo:
\begin{minted}[linenos]{arm}
cmp r2, r3   ; setta i flag
addeq r0, r0, r1   ; se i == j, salva in r0 g + h
subne r0, r0, r1   ; altrimenti, salva in r0 g - h
\end{minted}

\subsection*{Condizioni di disuguaglianza}
Vediamo ora invece una condizione di disuguaglianza:
\begin{minted}[linenos, breaklines, tabsize = 4]{c}
if (i < j){
	f = g + h;
} else {
	f = g - h;
}
\end{minted}
\subsubsection{Implementazione MIPS}
Assumendo che le variabili vengano mappate come sopra, il codice MIPS non appare troppo diverso, se non per l'utilizzo dell'istruzione \mintinline{asm}{slt}:
\begin{minted}[linenos, tabsize = 4]{asm}
  slt $a2, $a2, $a3   # setta $a2 se i < j
  beq $a2, $zero, L2   # se $a2 == 0 (non settato), vai a L2
  add $v0, $a0, $a1
  j L3   # salta a L3
L2:
  sub $v0, $a0, $a1
L3:
\end{minted}
\subsubsection{Implementazione x86}
L'implementazione è quasi identitica a quella della condizione di uguaglianza, cambia solamente l'istruzione di salto.
\begin{minted}[linenos, tabsize = 4]{gas}
  cmpq %rcx, %rdx  # confronta i e j e setta il flag
  jge L2   # se i < j, salta a L2
  leaq (%rdi, %rsi), %rax  # salva in %rax g + h
  jmp L3
L2:
  movq %rdi, %rax  # sposta g da %rdi a %rax
  subq %rsi, %rax  # salva in %rax
L3:
\end{minted}
Anche l'implementazione con \mintinline{gas}{cmov} è simile:
\begin{minted}[linenos, tabsize = 4]{gas}
leaq (%rdi, %rsi), %rax  # salva in %rax g + h
subq %rsi, %rax  # salva in %rax
cmpq %rcx, %rdx  # flag
cmovge %rdi, %rax  # sposta %rdi in %rax se il i < j
\end{minted}

\subsubsection{Implementazione ARM}
Ancora una volta, le indicazioni condizionali delle istruzioni ARM ci semplificano la vita:
\begin{minted}[linenos]{arm}
cmp r2, r3   ; setta i flag
addlt r0, r0, r1   ; se i < j, salva in r0 g + h
subge r0, r0, r1   ; altrimenti, salva in r0 g - h
\end{minted}

\section{Cicli}
Ora analizziamo l'implementazione di un ciclo con condizione di terminazione basata sui valori contenuti in un array di interi. Il codice C è il seguente:
\begin{minted}[linenos, breaklines, tabsize = 4]{c}
int i = 0;
while (a[i] == k){
	i += 1;
}
\end{minted}

\subsubsection{Implementazione MIPS}
In MIPS mappiamo \mintinline{c}{k} in \register{\$a0}, l'indirizzo di \mintinline{c}{a} in\register{\$a1} e \mintinline{c}{i} in \register{\$v0}.

\begin{minted}[linenos, tabsize = 4]{asm}
	move $v0, $zero	 # i = 0
L1:
	sll $t1, $v0, 2	 # salva in $t1 4 * i
	add $t1, $t1, $a1	 # ottieni l'indirizzo di a[i]
	lw $t0, 0($t1)	 # carica in $t0 il contenuto di a[i]
	bne $t0, $a0, L2	 # salta a l2 se a[i] è diverso da k
	addi $v0, $v0, 1
	j L1
L2:
\end{minted}

Implementazione di \register{gcc}:
\begin{minted}[linenos, tabsize = 4]{asm}
	lw $a2, 0($a1)
	bne $a2, $a0, L2
	add $a1, $a1, 4
	move $v0, $zero
L1:
	addi $a1, $a1, 4
	lw $v1, -4($a1)
	addi $v0, $v0, 1
	beq $v1, $a0, L1
	j L3
L2:
	move $v0, $zero
L3:
\end{minted}

\subsubsection{Implementazione x86}
\begin{minted}[linenos, tabsize = 4]{gas}
	cmpl (%rsi), %edi
	jne L2
	movq $0, %rax
L1:
	addq $1, %rax
	cmpl %edi, (%rsi, %rax, 4)
	je L1
	jmp L3
L2:
	movq $0, %rax
L3:
\end{minted}

\subsubsection{Implementazione ARM}
\begin{minted}[linenos,tabsize = 4]{arm}
	ldr r3, [r1]
	cmp r0, r3
	bne L2
	mov r3, #0
L1:
	add r3, r3, #1
	ldr r2 [r1, #4]!
	cmp r2, r0
	beq L1
	b L3
L2:
	mov r3, #0
L3:
	mov r0, r3
\end{minted}

\section{Invocazione di subroutine}
\subsection*{Funzioni foglia}

\begin{minted}[linenos,tabsize = 4]{c}
int esempio_foglia(int g, int h, int i, int j){
	int f;
	f = (g + h) - (i + j);
	return f;
}
\end{minted}

\subsubsection{Implementazione MIPS}
\begin{minted}[linenos,tabsize = 4]{asm}
esempio_foglia:
	addu $a0, $a0, $a1
	addu $v0, $a2, $a3
	subu $v0, $a0, $v0
	jr $ra
\end{minted}

\subsubsection{Implementazione x86}
\begin{minted}[linenos,tabsize = 4]{gas}
esempio_foglia:
	leal (%rdi, %rsi), %eax
	addl %ecx, %edx
	subl %edx, %eax
	ret
\end{minted}

\subsubsection{Implementazione ARM}
\begin{minted}[linenos,tabsize = 4]{arm}
esempio_foglia:
	add r0, r0, r1
	add r3, r2, r3
	rsb r0, r3, r0
	bx lr
\end{minted}

\subsection*{Funzioni non foglia}

\begin{minted}[linenos,breaklines, tabsize = 4]{c}
int inc(int n){
	return n + 1;
}

int f(int x){
	return inc(x) - 4;
}
\end{minted}

\subsubsection{Implementazione MIPS}
\begin{minted}[linenos, breaklines, tabsize = 4]{asm}
inc:
	addiu $v0, $a0, 1
	jr $ra

f:
	addiu $sp, $sp, -4
	sw $ra, 0($sp)
	jal inc
	addiu $v0, $v0, -4
	lw $ra, 0($sp)
	addiu $sp, $sp, 4
	jr $ra
\end{minted}

Implementazione \register{gcc}:
\begin{minted}[linenos, breaklines, tabsize = 4]{asm}
inc:
	addiu $v0, $a0, 1
	jr $ra

f:
	addiu $sp, $sp, -8
	sw $ra, 4($sp)
	jal inc
	addiu $v0, $v0, -4
	lw $ra, 4($sp)
	addiu $sp, $sp, 8
	jr $ra
\end{minted}

\subsubsection{Implementazione x86}
\begin{minted}[linenos, breaklines, tabsize = 4]{gas}
inc:
	leal 1(%rdi), %eax
	ret

f:
	call inc
	subl $4, %eax
	ret
\end{minted}

\subsubsection{Implementazione ARM}
\begin{minted}[linenos, breaklines, tabsize = 4]{arm}
inc:
	add r0, r0, 31
	bx lr

f:
	str lr, [sp, #-4]!
	bl inc
	sub r0, r0 #4
	ldr pc, [sp], #4
\end{minted}

Implementazione \register{gcc}:
\begin{minted}[linenos, breaklines, tabsize = 4]{arm}
inc:
	add r0, r0, #1
	bx lr

f:
	stmfd sp!, {r4, lr}
	bl inc
	sub r0, r0 #4
	ldmfd sp!, {r4, pc}
\end{minted}

\section{Ordinamento di array}
Mostriamo ora le diverse traduzioni della seguente funzione \emph{insert sort} scritta in C:

\begin{minted}[linenos, breaklines, tabsize = 4]{c}
void sposta(int v[], int i){
	int j;
	int appoggio;

	appoggio = v[i];
	j = i - 1;

	while((j >= 0) && (v[j] > appoggio)){
		v[j +	1] = v[j];
		j = j - 1;
	}

	v[j + 1]
}

void ordina(int v[], int n){
	int i = 1;

	while (i < n){
		sposta(v, i);
		i = i + 1;
	}
}
\end{minted}

\subsubsection{Implementazione MIPS}
Prima versione, "fatta a mano":
\begin{minted}[linenos, breaklines, tabsize = 4]{asm}
sposta:
		# a0 = v, a1 = i
		# t0 = j * 4, t1 = appoggio
	sll $t0, $a1, 2
	add $t2, $a0, $t0	 # t2 = &v[i]
	lw $t1, 0($t2)	 # appoggio = v[i]
	addi $t0, $t0, -4

ciclo:
	slt $t3, $t0, $zero	 # t3 = 1 se j < 0
	bne $t3, $zero, out	 # salta a out se j < 0
	add $a2, $a0, $t0	 # t2 = &v[j]
	lw $t4, 0($t2)	 # t4 = v[j]
	slt $t3, $t1 , $t4	 # t3=1 se t1<t4 (appoggio < v[j])
	beq $t3, $zero, out	 # esci se appoggio >= v[j]
	addi $t5, $t0, 4	 # t5 = (j + 1) * 4
	add $t2, $a0, $t5	 # t2 = &v[j + 1]
	sw $t4, 0($t2)
	addi $t0, $t0, -4
	j ciclo

out:
	addi $t5, $t0, 4
	add $t2, $a0, $t5	 # t2 = &v[j + 1]
	sw $t1, 0($t2)
	jr $ra

ordina:
		# a0 = v, a1 = n
		# s0 = i
	addi $sp, $sp, -12
	sw $s0, 0($sp)
	sw $ra, 4($sp)
	sw $a1, 8($sp)

	addi $s0, $zero, 1	 # i = 1

loop_ordina:
	slt $t0, $s0, $a1	 # t0 = 1 se i < n
	beq $t0, $zero, out_ordina
	add $a1, $s0, $zero
	jal sposta
	lw $a1, 8($sp)
	addi $s0, $s0, 1
	j loop_ordina

out_ordina:
	lw $s0, 0($sp)
	lw $ra, 4($sp)
	addi $sp, $sp, 12
	jr $ra
\end{minted}

Ecco invece l'implementazione, più complessa, che ci presenta il nostro fido \register{gcc}:

\begin{minted}[linenos, breaklines, tabsize = 4]{asm}
sposta:
	sll $v0, $a1, 2
	addu $v0, $a0, $v0
	lw $t0, 0($v0)
	addiu $v0, $a1, -1
	bltz $v0, L2
	move $a2, $v0
	sll $v1, $v0, 2
	addu $v1, $a0, $v1
	lw $v1, 0($v1)
	slt $a3, $t0, $v1
	beq $a3, $zero, L2
	sll $a1, $a1, 2
	addu $a1, $a0, $a1
	li $t1 , -1	 # 0xffffffffffffffff

L3:
	addiu $a2, $a2, 1
	sll $a2, $a2, 2
	addu $a2, $a0, $a2
	sw $v1, 0($a2)
	addiu $v0, $v0, -1
	beq $v0, $t1, L2
	move $a2, $v0
	addiu $a1, $a1, -4
	lw $v1, 4($a1)
	slt $a3, $t0, $v1
	bne $a3, $zero, L3

L2:
	addiu $v0, $v0, 1
	sll $v0, $v0, 2
	addu $a0, $a0, $v0
	sw $t0, 0($a0)
	jr $ra

ordina:
	addiu $sp, $sp, -16
	sw $ra, 12($sp)
	sw $s2, 8($sp)
	sw $s1, 4($sp)
	sw $s0, 0($sp)
	move $s1, $a1
	slt $v0, $a1, 2
	bne $v0, $zero, L5
	move $s2, $a0
	li $s0, 1	 # 0x1

L7:
	move $a0, $s2
	move $a1, $s0
	jal sposta
	addiu $s0, $s0, 1
	bne $s0, $s1, L7

L5:
	lw $ra, 12($sp)
	lw $s2, 8($sp)
	lw $s1, 4($sp)
	lw $s0, 0($sp)
	addiu $sp, $sp, 16
	jr $ra
\end{minted}

\subsubsection{Implementazione x86}
Ecco una possibile implementazione Intel, sempre artigianale:
\begin{minted}[linenos, breaklines, tabsize = 4]{gas}
sposta:
		# rdi = v, rsi = i
		# rax = j , r10d = appoggio
	movq %rsi, %rax
	movl (%rdi, %rax, 4), %r10d	 # appoggio = v[i]
	dec %rax

ciclo:
	cmpq $0, %rax	 # confronta 0 e rax
	jl out	 # esci se j < 0
	movl (%rdi, %rax, 4), %r11d	 # metti v[j] in %r11
	cmpl %r10d , %r11d	 # confronta v[j] e appoggio
	jle out	 # se v[j] < appoggio, esci
	movl %r11d, 4(%rdi, %rax  4)
	dec %rax
	jmp ciclo

out:
	movl %r10d, 4(%rdi, %rax, 4)
	ret

ordina:
		# rdi = v, rsi = n
		# rbx = i
	pushq %rbx
	movq $1, %rbx

loop_ordina:
	cmp %rbx , %rsi
	jle out_ordina
	pushq %rsi
	movq %rbx, %rsi
	call sposta
	popq %rsi
	inc %rbx
	jmp loop_ordina

out_ordina:
	popq %rbx
	ret
\end{minted}

Implementazione di \register{gcc}:

\begin{minted}[linenos, breaklines, tabsize = 4]{gas}
sposta:
	movslq %esi, %rax
	leaq 0(%rax, 4 ), %r8
	movl (%rdi, %rax, 4), %ecx
	subl $1, %esi
	js L2
	movslq %esi, %rdx
	movl -4(%rdi, %r8), %eax
	cmpl %eax, %ecx
	jge L2

L4:
	movl %eax, 4(%rdi, %rdx, 4)
	subl $1, %esi
	cmpl $-1, %esi
	je L2
	movslq %esi, %rdx
	movl (%rdi, %rdx, 4), %eax
	cmpl %eax, %ecx
	jl L4

L2:
	movslq %esi, %rsi
	movl %ecx, 4(%rdi, %rsi, 4)
	ret

ordina:
	cmpl $1, %esi
	jle L12
	pushq %r12
	pushq %rbp
	pushq %rbx
	movl %esi, %ebp
	movq %rdi, %r12
	movl $1, %ebx

L8:
	movl %ebx, %esi
	movq %r12, %rdi
	call sposta
	addl $1, %ebx
	cmpl %ebx, %ebp
	jne L8
	popq %rbx
	popq %rbp
	popq %r12

L12:
	ret
\end{minted}

\subsubsection{Implementazione ARM}
\begin{minted}[linenos, breaklines, tabsize = 4]{arm}
sposta:
	mov r2, r1, asl #2
	add r3 , r0, r2
	ldr ip, [r0, r1, asl #2]
	subs r1, r1, #1
	bmi L2
	ldr r2, [r3, #-4]
	cmp ip, r2
	bge L2

L3:
	str r2, [r3], #-4
	sub r1, r1, #1
	cmn r1, #1
	beq L2
	ldr r2, [r3, #-4]
	cmp ip, r2
	blt L3

L2:
	add r1, r1, #1
	str ip, [r0, r1, asl #2]
	bx lr

ordina:
	cmp r1, #1
	bxle lr
	stmfd sp!, {r4, r5, r6, lr}
	mov r5, r1
	mov r6, r0
	mov r4, #1

L8:
	mov r1, r4
	mov r0, r6
	bl sposta
	add r4, r4, #1
	cmp r5, r4
	bne L8
	ldmfd sp!, {r4, r5, r6, pc}
\end{minted}

\section{Funzione \(2^{n}\)}

\subsection*{Via iterativa}

\begin{minted}[linenos, breaklines, tabsize = 4]{c}
int potenza_due(int n){
	if (n == 0){
		return 1;
	} else {
		int ret = 1;
		int i;
		for (i=0; i<n; i++){
			ret = ret * 2;
		}

		return ret;
		}
}
\end{minted}

\subsubsection{Implementazione MIPS}

\begin{minted}[linenos, breaklines, tabsize = 4]{asm}
potenza_due:
	beq $a0, $zero, L4
	blez $a0, L5
	move $v1, $zero
	li $v0, 1	 # 0x1

L3:
	sll $v0, $v0, 1
	addiu $v1, $v1, 1
	bne $v1, $a0, L3
	jr $ra

L4:
	li $v0, 1	 # 0x1
	jr $ra

L5:
	li $v0, 1	 # 0x1
	jr $ra
\end{minted}

\subsubsection{Implementazione x86}

\begin{minted}[linenos, breaklines, tabsize = 4]{gas}
potenza_due:
	testl %edi, %edi
	jle L4
	movl $0, %edx
	movl $1, %eax

L3:
	addl %eax, %eax
	addl $1, %edx
	cmpl %edx, %edi
	jne L3
	ret

L4:
	movl $1, %eax
	ret
\end{minted}

\subsubsection{Implementazione ARM}

\begin{minted}[linenos, breaklines, tabsize = 4]{arm}
potenza_due:
	subs r2, r0, #0
	ble L4
	mov r3, #0
	mov r0, #1
L3:
	mov r0, r0, asl #1
	add r3, r3, #1
	cmp r2, r3
	bne L3
	bx lr
L4:
	mov r0, #1
	bx lr
\end{minted}

\subsection*{Via ricorsiva}

\begin{minted}[linenos, breaklines, tabsize = 4]{c}
int potenza_due (int n){
	if (n < 1){
	return 1;
	} else {
		return 2 * potenza_due(n - 1);
	}
}
\end{minted}

\subsubsection{Implementazione MIPS}

\begin{minted}[linenos, breaklines, tabsize = 4]{asm}
potenza_due:
	blez $a0, L3
	addiu $sp, $sp, -8
	sw $ra, 4($sp)
	addiu $a0, $a0, -1
	jal potenza_due
	sll $v0, $v0, 1
	j L2

L3:
	li $v0, 1	 # 0x1
	jr $ra

L2:
	lw $ra, 4($sp)
	addiu $sp, $sp, 8
	jr $ra
\end{minted}

\subsubsection{Implementazione x86}

\begin{minted}[linenos, breaklines, tabsize = 4]{gas}
potenza_due:
	movl $1, %eax
	testl %edi, %edi
	jle L6
	subq $8, %rsp
	subl $1, %edi
	call potenza_due
	addl %eax, %eax
	addq $8, %rsp

L6:
	ret
\end{minted}

\subsubsection{Implementazione ARM}

\begin{minted}[linenos, breaklines, tabsize = 4]{arm}
potenza_due:
	cmp r0, #0
	ble L3
	stmfd sp!, {r4, lr}
	sub r0, r0, #1
	bl potenza_due
	mov r0, r0, asl #1
	ldmfd sp!, {r4, pc}

L3:
	mov r0, #1
	bx lr
\end{minted}

\section{Copia stringa}
Consideriamo ora una funziona che copia un array di caratteri:
\begin{minted}[linenos, breaklines, tabsize = 4]{c}
void copia_stringa(char *d , const char *s){
	int i = 0;

	while((d[i] = s[i]) != 0){
		i+=1;
	}
}
\end{minted}

\subsubsection{Implementazione MIPS}
Abbiamo già discusso l'implementazione MIPS di questa funzione in 6.9, per cui qui riporteremo solamente il codice, generato da \register{gcc}:
\begin{minted}[linenos, breaklines, tabsize = 4]{asm}
copia_stringa:
	lb $v0, 0($a1)
	sb $v0, 0($a0)
	beq $v0, $zero, L5
	move $v0, $zero
L3:
	addiu $v0, $v0, 1
	addu $v1, $a1, $v0
	lb $v1, 0($v1)
	addu $a2, $a0, $v0
	sb $v1, 0($a2)
	bne $v1, $zero, L3
L5:
	jr $ra
\end{minted}

\subsubsection{Implementazione x86}

\begin{minted}[linenos, breaklines, tabsize = 4]{gas}
copia_stringa:
	movzbl (%rsi), %eax
	movb %al, (%rdi)
	testb %al, %al
	je L1
	movl $0, %eax
L3 :
	addl $1, %eax
	movslq %eax, %rcx
	movzbl (%rsi, %rcx), %edx
	movb %dl, (%rdi, %rcx)
	testb %dl, %dl
	jne L3
L1:
	ret
\end{minted}

\subsubsection{Implementazione ARM}

\begin{minted}[linenos, breaklines, tabsize = 4]{arm}
copia_stringa:
	ldrb r3, [r1]
	strb r3, [r0]
	cmp r3, #0
	bxeq lr
L3:
	ldrb r3, [r1, #1]!
	strb r3, [r0, #1]!
	cmp r3, #0
	bne L3
	bx lr
\end{minted}

\section{Successione di Fibonacci}
Presentiamo ora l'implementazione di una funzione che calcola l'\(n\)-esimo numero di Fibonacci:
\begin{minted}[linenos, breaklines, tabsize = 4]{c}
int fibonacci(int n){
	if (n < 2){
		return 1;
	} else {
		return 3 * fibonacci(n - 1) - fibonacci(n - 2);
	}
}
\end{minted}

\subsubsection{Implementazione MIPS}

\begin{minted}[linenos, breaklines, tabsize = 4]{asm}
fibonacci:
	addiu $sp, $sp, -16
	sw $ra, 12($sp)
	sw $s1, 8($sp)
	sw $s0, 4($sp)
	move $s0, $a0
	slt $v0, $a0, 2
	bne $v0, $zero, L3
	addiu $a0, $a0, -1
	jal fibonacci
	move $s1, $v0
	addiu $a0, $s0, -2
	jal fibonacci
	sll $v1, $s1, 1
	addu $s1, $v1, $s1
	subu $v0, $s1, $v0
	j L2
L3:
	li $v0, 1	 # 0x1
L2 :
	lw $ra, 12($sp)
	lw $s1, 8($sp)
	lw $s0, 4($sp)
	addiu $sp, $sp, 16
	jr $ra
\end{minted}

\subsubsection{Implementazione x86}

\begin{minted}[linenos, breaklines, tabsize = 4]{gas}
fibonacci:
	movl $1, %eax
	cmpl $1, %edi
	jle L6
	pushq %rbp
	pushq %rbx
	subq $8, %rsp
	movl %edi, %ebx
	leal -1(%rdi), %edi
	call fibonacci
	movl %eax, %ebp
	leal -2(%rbx), %edi
	call fibonacci
	leal 0(%rbp, %rbp, 2), %edx
	subl %eax, %edx
	movl %edx, %eax
	addq $8, %rsp
	popq %rbx
	popq %rbp
L6:
	ret
\end{minted}

\subsubsection{Implementazione ARM}

\begin{minted}[linenos, breaklines, tabsize = 4]{arm}
fibonacci:
	cmp r0, #1
	ble L3
	stmfd sp!, {r4, r5, r6, lr}
	mov r5, r0
	sub r0, r0, #1
	bl fibonacci
	mov r4, r0
	sub r0, r5, #2
	bl fibonacci
	add r4, r4, r4, lsl #1
	rsb r0, r0, r4
	ldmfd sp!, {r4, r5, r6, pc}
L3:
	mov r0, #1
	bx lr
\end{minted}

\section{Serie tail recursive}
Vediamo ora una funzione che calcola i primi \(n\) numeri naturali, in versione tail recursive:
\begin{minted}[linenos, breaklines, tabsize = 4]{c}
int sum(int n, int acc){
	if (n > 0){
		return sum(n - 1, acc + n);
	} else {
		return acc;
	}
}
\end{minted}

\subsection*{Senza ottimizzazione di chiamate in coda}
Vediamone inizialmente una traduzione fatta senza l'eliminazione delle chiamate ricorsive:
\subsubsection{Implementazione MIPS}

\begin{minted}[linenos, breaklines, tabsize = 4]{asm}
sum:
	move $v1, $a0
	move $v0, $a1
	blez $a0, L5
	addiu $sp, $sp, -8
	sw $ra, 4($sp)
	addiu $a0, $a0, -1
	addu $a1, $a1 , $v1
	jal sum
	lw $ra, 4($sp)
	addiu $sp, $sp,8
L5:
	jr $ra
\end{minted}

\subsubsection{Implementazione x86}

\begin{minted}[linenos, breaklines, tabsize = 4]{gas}
sum :
	movl %esi, %eax
	testl %edi, %edi
	jle L6
	subq $8, %rsp
	addl %edi, %esi
	subl $1, %edi
	call sum
	addq $8, %rsp
L6:
	ret
\end{minted}

\subsubsection{Implementazione ARM}

\begin{minted}[linenos, breaklines, tabsize = 4]{arm}
sum:
	cmp r0, #0
	ble L3
	stmfd sp!, {r4, lr}
	add r1, r0, r1
	sub r0, r0, #1
	bl sum
	ldmfd sp!, {r4, pc}
L3:
	mov r0, r1
	bx lr
\end{minted}

\subsection*{Chiamate in coda ottimizzate}
Proponiamo ora la traduzione che \register{gcc} esegue attivando il comando -\register{foptimize}-\register{sibling}-\register{calls}, la cui funzione è ottimizzare le chiamate in coda:
\subsubsection{Implementazione MIPS}

\begin{minted}[linenos, breaklines, tabsize = 4]{asm}
sum:
	move $v0, $a1
	blez $a0, L2
L4:
	addu $v0, $v0, $a0
	addiu $a0, $a0, -1
	bne $a0, $zero, L4
L2:
	jr $ra
\end{minted}

\subsubsection{Implementazione x86}

\begin{minted}[linenos, breaklines, tabsize = 4]{gas}
sum:
	movl %esi, %eax
	testl %edi , %edi
	jle L2
L4:
	addl %edi, %eax
	subl $1, %edi
	jne L4
L2:
	ret
\end{minted}

\subsubsection{Implementazione ARM}

\begin{minted}[linenos, breaklines, tabsize = 4]{arm}
sum:
	cmp r0, #0
	ble L2
L4:
	sub r3, r0, #1
	add r1, r1, r0
	mov r0, r3
	cmp r3, #0
	bne L4
L2:
	mov r0, r1
	bx lr
\end{minted}

\end{document}
